\documentclass[]{foi} 

\usepackage[utf8]{inputenc}
\usepackage{lipsum}
\usepackage{fvextra}
\usepackage{csquotes}
\usepackage{booktabs}
\usepackage{caption}
\usepackage{array}
\usepackage{float} 
\usepackage{makecell}
\usepackage{graphicx}



\vrstaRada{\diplomski}

\title{Informativna platforma s inteligentnim asistentom o Erasmus+ iskustvu u \L\'{o}d\'{z}u.}

\predmet{}

\author{Aleksandar Babić} 

\spolStudenta{\musko} 

\mentor{Bogdan Okreša Đurić}

\spolMentora{\musko} 

\titulaProfesora{doc.~dr.~sc.}

\godina{2025}
\mjesec{rujan}

\indeks{0016147642}

\smjer{Baze podataka i baze znanja}

\sazetak{Teoretski dio rada istražuje studiranje kao važan segment u životnom razdoblju pojedinca te se dotiče razmjene studenata u svrhu Erasmus+ studijskog boravka. Teoretski dio završava poglavljem posvećenim inteligentnim asistentima temeljenima na RAG (eng. Retrieval-Augmented Generation) tehnologiji.
Praktični dio rada obuhvaća izradu web-stranice s tematikom Erasmusa+ studijskog boravka i izradu inteligentnog asistenta, pri čemu će se iskoristiti postojeći jezični model. To podrazumijeva treniranje i prilagođavanje specifičnom skupu podataka, kako bi se osiguralo da inteligentni asistent odgovara upitima korisnika i pruža relevantne informacije.
}

\kljucneRijeci{Inteligentni asistent, RAG, web stranica, studiranje, Erasmus+, umjetna inteligencija.}

\acrodef{VAS}{višeagentni sustav}


\begin{document}

\maketitle

\tableofcontents

\makeatletter \def\@dotsep{4.5} \makeatother
\pagestyle{plain}



\chapter{Uvod}

uvod



\chapter{Studiranje kao dio života pojedinca}

\section{Odluka i benefiti}
\section{Erasmus+ studenska razmjena}
\subsection{Izazovi}
\subsection{Kako se izboriti?}
\subsection{Kultura kao faktor u akademskom uspjehu}



\chapter{Aktualne tehnologije i alati za uspješno snalaženje i informiranje}
\section{Važnost inteligentnih asistenata}

Davne 1961. godine IBM je predstavio prvi sustav koji je mogao prepoznati govor (izgovorene znamenke). Do 1990-ih kreirani su prvi komercijalni 
osobni asistentni koji su se aktivirali na postojanje glasa. Godine 2010. svijet je prvi put čuo za Sirija, Appleovog osobnog asistenta koji je 
pokrenutna temelju umjetne inteligencije. U to vrijeme Siri je predstavljao revolucionarno rješenje. Petnaest godina kasnije, korisnici su 
zatrpani s ogromnom količinom inteligentnih asistenata. Korisničko iskustvo (eng. UX) je sada više nego ikad važno. Održavanje zadovoljstva 
korisnika/kupca i upravljanje njihovim očekivanjima zahtijeva stalne inovacije i poboljšanja. Korisnici žele da stranice budu intuitivne i učinkovite
te da na neki način zadovolje njihove potrebe. AI asistentni i chatbotovi su postali popularni alati koji pomažu u poboljšanju korisničkog iskustva.
Točnije rečeno dodaju novi način interakcije između korisnika i pružatelja usluge i samim time povećavaju angažman, zadovoljstvo ili profit.
Prednosti inteligentnih asistenata uključuju \cite{buchan2024ai}:
\begin{itemize}
    \item \textbf{Brzina i jednostavnost} - AI asistentni omogućuju korisnicima da brzo i jednostavno dobiju ono što im je potrebno, smanjujući
    gubitak vremena. Tik Tok je dobar primjer kako usluga može neindirektno zadržati korisnika na njihovoj platformi jer su razvili mehanizam koji
    je naučio pretpostavljati što korisnik želi gledati.
    \item \textbf{Pružanje podrške 24/7} - Ai asistentni su dostupni stalno, čime je omogućena podrška korisnicima kad god je to potrebno. 
    \item \textbf{Prihvaćanje od strane ljudi} - razgovor s AI asistentima se odvija kroz razne dizajnove korisničkih sučelja, gdje korisnici mogu
    komunicirati na ljudski razumljiv način s asistentima. 
\end{itemize}

U sljedećem poglavlju je prikazano kako se mogu kreirati inteligentni asistentni temeljeni na RAG (eng. Retrieval-Augmented generation) tehnologiji.
\newpage
\section{RAG tehnologija}

RAG je proces optimizacije izlaza velikog jezičnog modela pri čemu se referencira na autorativnu bazu znanja izvan izvora podataka za obuku 
prije generiranje odgovora. Veliki jezični modeli (eng. LLM) obučavaju se na ogromnim količinama podataka i koriste milijarde parametara za 
generiranje izlaza za zadatke poput odgovaranja pitanja, prevođenja jezika i dovršavanja rečenica. Ideja RAG-a je da proširi već močne mogućnosti 
LLM-ova na određene specifične domene, a sve to bez potrebe za ponovnom obukom modela. Takav pristup poboljšava LLM-ove u smislu točnosti, 
korisnosti i relevantnosti \cite{awsRAG2025}.

LLM-ovi su ključna tehnologija umjetne inteligencije (AI) koja pokreće chatbotove. Priroda LLM-ova je takva da ponekad uvode dozu nepredvidljivosti u
u generiranje odgovora. Osim toga, podaci o LLM obuci su statični i imaju kranji rok za znanje koje posjeduju. Negativne strane LLM-ova uključuju \cite{awsRAG2025}:

\begin{itemize}
    \item {Iznošenje lažnih informacija kada na njih nema odgovora,} 
    \item {Prikazivanje zastarjelih informacija kada korisnik očekuje aktualan odgovor,} 
    \item {Izrada odgovora iz neautoriziranih izvora,} 
    \item {Stvaranje netočnih odgovora zbog terminološke zbrke, različiti izvori obuke koriste istu terminologiju za opisivanje različite stvari}
\end{itemize}

Model velikog jezika može se zamisliti kao enuzijastičnog zaposlenika koji odbija biti informiran o aktualnim događajima, ali uvijek odgovara
na pitanja s visokim samopouzdanjem. Nažalost, takav stav može negativno utjecati na povjerenje korisnika i nije nešto što bi trebalo postojati
kod chatbotova. Iz takvih razloga RAG tehnologija preusmjerava LLM kako bi dohvatio relevantne informacije iz unaprijed definiranih izvora znanja \cite{awsRAG2025}.
\newpage

Bez RAG-a, LLM uzima korisnički unos i stvara odgovor na temelju informacija na kojima je obučen. S RAG-om se uvodi komponenta za pronalaženje 
infromacija koja koristi korisnički unos kako bi prvo iuvikla informacije iz novog izvora podataka. Nakon toga LLM kombinira novo znanje i svoje 
podatke za stvaranje boljih odgovora \cite{awsRAG2025}.

\begin{figure}[h!]
    \centering
    \includegraphics[width=0.8\textwidth]{./assets/Konceptualni tok korištenja RAG-a s LLM-ovima.jpg} 
    \caption{Konceptualni tok korištenja RAG-a s LLM-ovima}
    \label{fig:slika1}
\end{figure}

U nekoliko koraka pojašnjeno je kako funkcioniraju stvari na prikazanoj slici \ref{fig:slika1}\cite{awsRAG2025}:

\begin{enumerate}
    \item \textbf{Stvaranje vanjskih podataka} - novi podaci izvan orginalnog skupa podataka koji su se koristili za treniranje LLM-a se zovu 
    vanjski podaci (eng. external data). Oni dolaze iz više izvora podataka kao što su: API-ji ili baze podataka. Podaci mogu postojati u različitim formatima.
    Jedna od tehnika (eng. embeeding language models) pretvara podatke u numerički oblik i sprema ih u vektorsku bazu podataka 
    (kako bi generativni modeli umjetne inteligencije to mogli razumjeti).
    \item \textbf{Preuzimanje relehantnih informacija} - Korisnički upit pretvara se u vektorski prikaz i usporešuje se s vektoriskim bazama podataka.
    Npr. ako korisnik postavi pitanje chatbotu "Koliko dana godišnjeg odmora imam?", chatbot ce morati preuzeti dokumente o politici godišnjeg 
    odmora uz evidenciju prošlih odmora za pojedinog zaposlenika. Takvi dokumenti će biti vraćeni korisniku jer su relevatni za njegov upit.
    \item \textbf{Proširivanje LLM upita} - Zatim, RAG model proširuje korisnički upit dodavanjem relevatnih dohvaćenih podataka u konteksu. 
    Prošireni upit omogućuje modelima velikih jezika generiranje točnih odgovora na korisničke upite.
    \item \textbf{Ažuriranje vanjskih podataka} - Vrlo je bitno održati ažurne informacije i to se može ostvariti automatiziranjem procesa u stvarnom vremenu ili periodičnoj obradi. 
\end{enumerate}

\newpage





\begin{table}[h!]
    \centering
    \caption{Prednosti RAG-a}
    \begin{tabular}{|>{\centering\arraybackslash}m{5cm}|p{10cm}|}
      \hline
      \textbf{Prednost} & \textbf{Opis} \\
      \hline
      Isplativa implementacija & Razvoj chatbota započinje korištenjem temeljnog modela (eng. Foundation models). Temeljni modeli su obično dostupni putem API ključeva i obučeni su na širokom spektru generaliziranih i neoznačenih podataka. Računalni i financijski troškovi za treniranje takvih modela su jako visoki. \\
      \hline
      Trenutne informacije & Iako su izvorni podaci LLM-a relevantni za specifične potrebe, teško je održati relevantnost. RAG se može koristiti za izravno povezivanje LLM-a s feedovima društvenih medija, web stranicama s vijestima i samim time pružiti korisnicima najnovije informacije. \\
      \hline
      Povećano povjerenje korisnika & RAG omogućuje prikazivanje točnih informacija s navođenjem izvora ili citata. Korisnici također mogu pretraživati izvorne dokumente ako im je potrebno dodatno pojašnjenje. \\
      \hline
      Više kontrole za razvojne programere & Učinkovitije testiranje i poboljšavanje aplikacija. Kontroliranje i mijenjanje izvora informacija. Rješavanje i ispravljanje problema. \\
      \hline
    \end{tabular}
\end{table}


\section{Integriranje inteligentnih asistenata u aplikacije i akademske sustave}




\chapter{Praktični dio rada}



\section{Izrada web stranice}

\section{Izrada inteligentnog asistenta}




\chapter{Zaključak}

zakljucak 

\makebackmatter
% generira popis korištene literature, popis slika (ako je primjenjivo), popis tablica (ako je primjenjivo) i popis isječaka koda (ako je primjenjivo)





\end{document}
