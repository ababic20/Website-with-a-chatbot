\documentclass[]{foi} 

\usepackage[utf8]{inputenc}
\usepackage{lipsum}
\usepackage{fvextra}
\usepackage{csquotes}


\vrstaRada{\diplomski}

\title{Informativna platforma s inteligentnim asistentom o Erasmus+ iskustvu u \L\'{o}d\'{z}u.}

\predmet{}

\author{Aleksandar Babić} 

\spolStudenta{\musko} 

\mentor{Bogdan Okreša Đurić}

\spolMentora{\musko} 

\titulaProfesora{doc.~dr.~sc.}

\godina{2025}
\mjesec{rujan}

\indeks{0016147642}

\smjer{Baze podataka i baze znanja}

\sazetak{Teoretski dio rada istražuje studiranje kao važan segment u životnom razdoblju pojedinca te se dotiče razmjene studenata u svrhu Erasmus+ studijskog boravka. Teoretski dio završava poglavljem posvećenim inteligentnim asistentima temeljenima na RAG (eng. Retrieval-Augmented Generation) tehnologiji.
Praktični dio rada obuhvaća izradu web-stranice s tematikom Erasmusa+ studijskog boravka i izradu inteligentnog asistenta, pri čemu će se iskoristiti postojeći jezični model. To podrazumijeva treniranje i prilagođavanje specifičnom skupu podataka, kako bi se osiguralo da inteligentni asistent odgovara upitima korisnika i pruža relevantne informacije.
}

\kljucneRijeci{Inteligentni asistent, RAG, web stranica, studiranje, Erasmus+, umjetna inteligencija.}

\acrodef{VAS}{višeagentni sustav}


\begin{document}

\maketitle

\tableofcontents

\makeatletter \def\@dotsep{4.5} \makeatother
\pagestyle{plain}



\chapter{Uvod}

uvod



\chapter{Metode i tehnike rada}

metode



\chapter{Razrada teme}





\section{Poglavlje druge razine }





\subsection{Poglavlje treće razine}




\subsection{Poglavlje četvrte razine}





\chapter{Praktični dio rada}




\section{Upute za oblikovanje izgleda rada}

\subsection{Oblikovanje stranice}



\chapter{Zaključak}

zakljucak 

\makebackmatter
% generira popis korištene literature, popis slika (ako je primjenjivo), popis tablica (ako je primjenjivo) i popis isječaka koda (ako je primjenjivo)





\end{document}
